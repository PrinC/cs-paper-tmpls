% Thanks IEEE and Michael Shell
\documentclass[10pt,journal,compsoc]{IEEEtran}
% Add the compsoc option for Computer Society conferences.
\usepackage[nocompress]{cite}

% if then else
\usepackage{etoolbox}

% floats and fancy boxes
\usepackage[caption=false,font=footnotesize,labelfont=sf,textfont=sf]{subfig}
\usepackage{fancyvrb}
\usepackage{multirow}
\usepackage{tabularx}
\usepackage{arydshln}

% babel
\usepackage[english]{babel}

% math symbols
\usepackage{amsmath}
\DeclareMathOperator*{\eq}{==}
\usepackage{amssymb}
\usepackage{ulem}
\normalem
\usepackage{array}
\usepackage{graphicx}

% for better ending page formating
\usepackage{balance}

% for pretty-printing URLs
\usepackage[hyphens]{url}

% algorithm blocks
\usepackage{algorithm}
\usepackage{algorithmicx}
\usepackage{algpseudocode}
\renewcommand{\algorithmicrequire}{\textbf{Input:}}
\renewcommand{\algorithmicensure}{\textbf{Output:}}

% colors
\usepackage{color}
\definecolor{javared}{rgb}{0.6,0,0} % for strings
\definecolor{javagreen}{rgb}{0.25,0.5,0.35} % comments
\definecolor{javapurple}{rgb}{0.5,0,0.35} % keywords
\definecolor{javadocblue}{rgb}{0.25,0.35,0.75} % javadoc

% code snippets
\usepackage{listings}
% \renewcommand{\lstlistingname}{Snippet}
\lstset{language=Java,
basicstyle=\ttfamily \footnotesize,
columns=fullflexible,
keywordstyle=\color{javapurple}\bfseries,
stringstyle=\color{javared},
commentstyle=\color{javagreen},
morecomment=[s][\color{javadocblue}]{/**}{*/},
stepnumber=1,
%numbers=left,
numbersep=3pt,
tabsize=4,
showspaces=false,
showstringspaces=false,
captionpos=b
}

\newcommand\MYhyperrefoptions{bookmarks=true,bookmarksnumbered=true,
pdfpagemode={UseOutlines},plainpages=false,pdfpagelabels=true,
colorlinks=true,linkcolor={black},citecolor={black},urlcolor={black},
pdftitle={Bare Demo of IEEEtran.cls for Computer Society Journals},%<!CHANGE!
pdfsubject={Typesetting},%<!CHANGE!
pdfauthor={Michael D. Shell},%<!CHANGE!
pdfkeywords={Computer Society, IEEEtran, journal, LaTeX, paper,
             template}}%<^!CHANGE!

% my favs
\newcommand{\p}{Project\-Name}
\newcommand{\cameraready}[2]{#1} % change to #2 for camera ready
\newcommand{\comments}[1]{}
\newcommand{\fixme}[1]{{\bf FIXME: #1}}

\makeatletter
\let\@copyrightspace\relax
\makeatother

% correct bad hyphenation here
\hyphenation{op-tical net-works semi-conduc-tor}


\begin{document}
%
% paper title
% can use linebreaks \\ within to get better formatting as desired
\title{Bare Demo of IEEEtran.cls for Conferences}


% author names and affiliations
% use a multiple column layout for up to three different
% affiliations
\author{Michael~Shell,~\IEEEmembership{Member,~IEEE,}
        John~Doe,~\IEEEmembership{Fellow,~OSA,}
        and~Jane~Doe,~\IEEEmembership{Life~Fellow,~IEEE}% <-this % stops a space
\IEEEcompsocitemizethanks{\IEEEcompsocthanksitem M. Shell is with the Department
of Electrical and Computer Engineering, Georgia Institute of Technology, Atlanta,
GA, 30332.\protect\\
% note need leading \protect in front of \\ to get a newline within \thanks as
% \\ is fragile and will error, could use \hfil\break instead.
E-mail: see http://www.michaelshell.org/contact.html
\IEEEcompsocthanksitem J. Doe and J. Doe are with Anonymous University.}% <-this % stops a space
\thanks{Manuscript received April 19, 2005; revised September 17, 2014.}}

\markboth{Journal of \LaTeX\ Class Files,~Vol.~13, No.~9, September~2014}%
{Shell \MakeLowercase{\textit{et al.}}: Bare Advanced Demo of IEEEtran.cls for Journals}

\IEEEtitleabstractindextext{%
\begin{abstract}
The abstract goes here.
\end{abstract}

% Note that keywords are not normally used for peerreview papers.
\begin{IEEEkeywords}
Computer Society, IEEEtran, journal, \LaTeX, paper, template.
\end{IEEEkeywords}}

% make the title area
\maketitle

\IEEEpeerreviewmaketitle


\IEEEraisesectionheading{\section{Introduction}\label{sec:introduction}}
This demo file is intended to serve as a ``starter file''
for IEEE conference papers produced under \LaTeX\ using
IEEEtran.cls version 1.7 and later.
% You must have at least 2 lines in the paragraph with the drop letter
% (should never be an issue)
I wish you the best of success.

\hfill mds
 
\hfill January 11, 2007

Packages could be found at CTAN~\cite{greenwade93}.
\subsection{Subsection Heading Here}
Subsection text here.


\subsubsection{Subsubsection Heading Here}
Subsubsection text here.


\section{Conclusion}
The conclusion goes here.

\appendices

\cameraready{}{
\section*{Acknowledgments}
The authors would like to thank...
}

% I normally balance the pages before bibs
\balance
\bibliographystyle{IEEEtran}
\bibliography{example} % put real bibs

% biography section
% 
% If you have an EPS/PDF photo (graphicx package needed) extra braces are
% needed around the contents of the optional argument to biography to prevent
% the LaTeX parser from getting confused when it sees the complicated
% \includegraphics command within an optional argument. (You could create
% your own custom macro containing the \includegraphics command to make things
% simpler here.)
%\begin{IEEEbiography}[{\includegraphics[width=1in,height=1.25in,clip,keepaspectratio]{mshell}}]{Michael Shell}
% or if you just want to reserve a space for a photo:

\begin{IEEEbiography}{Michael Shell}
Biography text here.
\end{IEEEbiography}

% if you will not have a photo at all:
\begin{IEEEbiographynophoto}{John Doe}
Biography text here.
\end{IEEEbiographynophoto}

% insert where needed to balance the two columns on the last page with
% biographies
%\newpage

\begin{IEEEbiographynophoto}{Jane Doe}
Biography text here.
\end{IEEEbiographynophoto}

% that's all folks
\end{document}
